\documentclass[sigplan,screen]{acmart}

\def \BibTeX{{\rm B\kern-.05em{\sc i\kern-.025em b}\kern-.08emT\kern-.1667em\lower.7ex\hbox{E}\kern-.125emX}}

\begin{document}

%TODO: a title?
\title{Assignment 1}

%
% Authors
%
\author{John Alden}
\affiliation{\institution{Arizona State University}}
\email{jzalden@asu.edu}

\author{Debarati Bhattacharyya}
\affiliation{\institution{Arizona State University}}
\email{dbhatt14@asu.edu}

%
% Abstract
%
\begin{abstract}
TODO?
\end{abstract}

%
% Title and Authors
%
\maketitle

%
% Paper Sections
%
\section{Introduction}
Often times, a project becomes big enough that it is infeasible for it to be worked on by a simgle Agile Team. Examples of such can be entire product suites or multi-departmental projects. Under these circumstances, it becomes more useful to divide the project amongst multiple agile teams. By the nature of the agile workflow, this can become chaotic and counter-productive unless managed properly. To ensure efficient and functional multi-team agile development process, teams should prioritize communication, clear objective definitions, cross-functional teams, and a decentralized but cohesive architecture.

\section{Multi-Team dynamics}
\subsection{Proper Communication in Agile}
Agile, normally being a chaotic and hasty managment process is difficult to coordinate with other agile processes. It is for the reason it is important to prioritize communication and documentation as this allows other teams to stay up to date and on a similar track. Without an emphasis on communication and documentation, by the nature of agile, projects tend to diverge.

Team communication in agile normally manifests itself in the form of the daily standup meeting. to facilitate cross-team communication, these standup meetings should be adjusted to contain all the teams. This promotes dicusssions on topics pertinent to more than 1 team. It is also important to make sure the the scrum masters are communicating constantly as communication desynchronization give way to project desynchronization. The agile alliance also defines a tactis known as "The Scrum of Scrums" where the Scrum masters from each team may go speak on the team's behalf. This allows scrum masters to stay up to date with the progress and work done by other teams.

\subsection{Objective Definition}
Another factor to be mindful of when facilitating work between multiple agile teams is to mindful that the goal of the project is clearly defined. This can also manifest in the form of making sure product owner meetings are conducted  with members from every team. Ensuring that every team has a clearly defined objective helps each team to be working towards a common goal. Even slight deviation in the project goal or suggestions from the sponsor could cause teams to create incompatible progress on the project.

\subsection{Cross-functional Teams}

\section{Design and Architecture}
TODO

\section{Conclusion}
TODO

\bibliographystyle{ACM-Reference-Format}
\bibliography{paper.bib}

\end{document}
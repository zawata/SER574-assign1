\documentclass[sigplan,screen]{acmart}

\def \BibTeX{{\rm B\kern-.05em{\sc i\kern-.025em b}\kern-.08emT\kern-.1667em\lower.7ex\hbox{E}\kern-.125emX}}

\begin{document}

%TODO: a title?
\title{Assignment 1}

%
% Authors
%
\author{John Alden}
\affiliation{\institution{Arizona State University}}
\email{jzalden@asu.edu}

\author{Debarati Bhattacharyya}
\affiliation{\institution{Arizona State University}}
\email{dbhatt14@asu.edu}

%
% Abstract
%
\begin{abstract}
TODO?
\end{abstract}

%
% Title and Authors
%
\maketitle

%
% Paper Sections
%
\section{Introduction}
Often times, a project becomes big enough that it is infeasible for it to be worked on by a simgle Agile Team.
Examples of such can be entire product suites or multi-departmental projects.
Under these circumstances, it becomes more useful to divide the project amongst multiple agile teams.
By the nature of the agile workflow, this can become chaotic and counter-productive unless managed properly.
To ensure efficient and functional multi-team agile development process, teams should prioritize communication, clear objective definitions, cross-functional teams, and a decentralized but cohesive architecture.

\section{Proper Communication in Agile}
TODO

\section{Object Definition}
TODO

\section{Cross-functional Teams}
TODO

\section{Design annd Architecture}
TODO

\section{Conclusion}


\bibliographystyle{ACM-Reference-Format}
\bibliography{paper.bib}

\end{document}
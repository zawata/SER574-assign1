\documentclass[sigplan,screen]{acmart}

\def \BibTeX{{\rm B\kern-.05em{\sc i\kern-.025em b}\kern-.08emT\kern-.1667em\lower.7ex\hbox{E}\kern-.125emX}}

\begin{document}

%
% Title
%
\title{Collaboration among Agile teams : A Few Factors}

%
% Authors
%
\author{John Alden}
\affiliation{\institution{Arizona State University}}
\email{jzalden@asu.edu}

\author{Debarati Bhattacharyya}
\affiliation{\institution{Arizona State University}}
\email{dbhatt14@asu.edu}

%
% Abstract
%
\begin{abstract}
This paper will discuss what practices are most conducive to collaboration multiple, individual Agile Teams. Among these practices, are giving particular thought to the dynamics of the Agile teams and how they should interact and communicate. Various tools have been developed to not only aid the communication between teams to aid specifically in the communication between agile teams. Adding to this, Important thought must be given to developing a decentralized and modular architecture to ensure and even work split.
\end{abstract}

%
% Make Title and Authors
%
\maketitle

%
% Paper Sections
%
\section{Introduction}
Often times, a project becomes big enough that it is infeasible for it to be worked on by a simple Agile Team. Examples of such can be entire product suites or multi-departmental projects. Under these circumstances, it becomes more useful to divide the project amongst multiple agile teams. By the nature of the agile workflow, this can become chaotic and counter-productive unless managed properly. To ensure efficient and functional multi-team agile development process, teams should prioritize communication, clear objective definitions, cross-functional teams, and a decentralized but cohesive architecture.

\section{Multi-Team dynamics}
\subsection{Proper Communication in Agile}
Agile, normally being a chaotic and hasty management process is difficult to coordinate with other agile processes. It is for the reason it is important to prioritize communication and documentation as this allows other teams to stay up to date and on a similar track. Without an emphasis on communication and documentation, by the nature of agile, projects tend to diverge.

Team communication in agile normally manifests itself in the form of the daily stand-up meeting. to facilitate cross-team communication, these stand-up meetings should be adjusted to contain all the teams. This promotes discussions on topics pertinent to more than 1 team. It is also important to make sure the the scrum masters are communicating constantly as communication desynchronization give way to project desynchronization. The agile alliance also defines a tactics known as "The Scrum of Scrums" where the Scrum masters from each team may go speak on the team's behalf. This allows scrum masters to stay up to date with the progress and work done by other teams.

\subsection{Objective Definition}
Another factor to be mindful of when facilitating work between multiple agile teams is to mindful that the goal of the project is clearly defined. This can also manifest in the form of making sure product owner meetings are conducted  with members from every team. Ensuring that every team has a clearly defined objective helps each team to be working towards a common goal. Even slight deviation in the project goal or suggestions from the sponsor could cause teams to create incompatible progress on the project.

\subsection{Cross-functional Teams}
It is important to make sure members of teams aren't specialized to the work they are doing. Agile teams should be multi-disciplinary and multi-functional to encourage communication and allow for diversity in work. Agile as a process encourages diversity in teams to allow for a smooth uniform workflow and encouraging specialists goes against this workflow. This becomes especially destructive to communication between teams as specialization leads to close-mindedness.

\subsection{Tools for Agile Development}
Tool are incredibly important in an agile project and become essential in a multi-team settings. The more people in an agile project, the more difficult it becomes to keep everything organized, up to date, and correct between several teams. Various tools have been developed to assist not just individual teams, but assist teams in working together to track progress, facilitate communication and present documentation.

\subsubsection{Tracking Progress}
Computerized Agile tools have been around since the advent of Agile, or computers, whichever came first. Some are better than others but when it comes to multi-team agile workflows, to be better, they need to accomodate and support multiple teams. One particularly amazing tool is called Jira which can be used with plugins to be useful for large teams to organize progress on a project

\subsubsection{Communication}
Communication is always important and becomes ever-moreso in circumstances where collaborating agile teams may not be in the same location. In this circumstance it becomes essential to have more than one communication tool as they would each serve a separate purpose. For day-to-day chatting and discussions, slack comes highly recommended by hundreds of thousands. For video and voice chatting services many options exist such Google Hangouts, Cisco's WebEx, LogMeIn's GoToMeeting, and Microsoft's Skype for business, each with their own pros and cons and each serving a similar purpose.

\subsubsection{Documentation}
Just as with communication, documenting progress or project information becomes increasingly necessary when the teams are distributed. Microsoft has a slough of tools designed expressly for this purpose such as OneDrive for Business, Sharepoint, Exchange, and many others. Each of these services allow for document storage and presentation. OneDrive even has initial workings to allow for collaborative document editing via it's Word, Excel, and PowerPoint Online tools. Much smaller and more open projects typically use Google Drive to do document storage and collaborative editing as it easy to learn, slim, and, more importantly, free.

\section{Design and Architecture}
The Agile manifesto emphasizes face to face communication as the most effective way of conveying information and collaboration between Agile teams. Effective collaboration is the key to a successful software development and product delivery for any Agile team. However, in the real scenario, this is constantly challenged due to the fact that teams are large and often distributed in the organizations, forcing communications to take place in different directions.
\par
Agile teams should be self-organizing to be able to produce the best architectures, requirements, and designs. To some extent, it is the responsibility of the larger organization to enable the collaborative environment for its teams for emergent architecture designs while providing guides and architectural standards for the entire organization system. A simple, easily implementable, collaborative architecture enables teams to enhance design, performance, usability and of cross-team implementation synchronization. Thus, it reaffirms the importance of design and architecture in the collaboration of Agile teams.

\subsection{Agile Design}
Design is an important and crucial part of any software project. Agile teams often struggle with when it comes to working collaboratively in coming up with a good design for a software mainly due to these factors:
\begin{itemize}
    \item Often teams focus on high-fidelity designs during the planning process, which results a waterfall culture throughout implementation
    \item Designers are shared across teams and have limited time to work with a particular team
    \item The lack of a proper system to report feedback to the engineering team
    \item Lack of clarity in separation in the code base between the logic and presentation layers making style changes difficult
\end{itemize}
Iterating on product design won't yield great results if it is not collaborative. Seeking the perspectives of the customers and developers at the outset of a project will help with the first design and guide the subsequent iterations. The product owner and the designer must spend time brainstorming and iterating on the product plan in the beginning to validate the requirements and make sure the engineering team’s time is well spent in solving actual problems that the customers face.Next they will begin to engage with the development team in an iterative fashion; decide the most important problem to solve, and add just enough design (and code) to get feedback on the solution. As the team engages in sprint planning and backlog reduction, the designers must be involved to make decisions about the product’s future direction. Keeping in mind that, Agile is about welcoming change, the design should be flexible enough to adapt to new trends and modern technologies.

\subsection{Agile Architecture}
Similar to Design, architecture must be created in an iterative manner, beginning with enough definition for the team to start development. It must guide the team in a direction that will give them the best chance for success, without being hindrance to the flexibility to build the software in an agile manner. Thus there should be a trade-off between allowing a team to choose what makes sense for their project while ensuring they don't pick technologies that are incompatible with the rest of the enterprise. Hence, the team builds software that aligns to the overall enterprise architecture without forcing the team to use an architecture that won’t be effective for their project.An agile approach to architecture relies on doing enough definition up front to get started, gathering feedback as we go, adjusting as needed, and iterating frequently to keep architecture and design in sync with the emerging application. For the first sprint, a high-level enterprise or system architecture should be created and agreed upon. For each subsequent sprint, team-based design is used to update the as-is design from the last sprint to account for feature additions and enhancements that will be made in the current sprint. As stories are tasked out for development and testing the team will use this new 'to be' design to help determine what will need to be implemented in order to satisfy the expected design.

\section{Conclusion}
Design and Architecture are two primary building blocks of any software system. While following a particular software development methodology, care should be taken so as not to compromise with respect to the quality of them. Thus, it is imperative that the team works collaboratively to come up with the design and architecture of a product.
https://www.agileconnection.com/article/agile-approach-software-architecture
https://www.atlassian.com/agile/design
https://tech.gsa.gov/guides/Collaboration_Across_Agile_Teams/
\bibliographystyle{ACM-Reference-Format}
\bibliography{paper.bib}


\end{document}